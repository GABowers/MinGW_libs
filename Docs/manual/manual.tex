% !TEX TS-program = xelatex % !TEX encoding = UTF-8

\documentclass[12pt]{achemso} % use larger type; default would be 10pt
\SectionNumbersOn

\usepackage{fontspec}
\defaultfontfeatures{Mapping=tex-text} 
\usepackage{xunicode} 
\usepackage{textcomp}

\setmainfont{DejaVu Serif}
\setsansfont{DejaVu Sans} 
\setmonofont{DejaVu Sans Mono}

\usepackage{geometry} 
\geometry{a4paper} 
\usepackage[parfill]{parskip}

\usepackage{graphicx} 
\usepackage{achemso} 
\mciteErrorOnUnknownfalse

\title{Vespucci: A free, cross-platform tool for spectroscopic imaging}
\author{Daniel P. Foose} \affiliation{Wright State University}
\email{foose.3@wright.edu} \date{}

\begin{document} 
\maketitle
\newpage \begin{center} Copyright \textcopyright \ 2015 Wright State University.
\end{center}

Permission is granted to copy, distribute and/or modify this document under the
terms of the GNU Free Documentation License, Version 1.3 or any later version
published by the Free Software Foundation; with no Invariant Sections, no
Front-Cover Texts, and no Back-Cover Texts. A copy of the license is availible
from the Free Software Foundation at \url{http://www.gnu.org/licenses/fdl.html}.

Please cite this documentation as a book if you use Vespucci in a published
work. A BibTeX citation follows: \cite{Vespucci} \begin{verbatim}
@Book{Vespucci, 
Title = {Vespucci: a free, cross-platform tool for spectroscopic
imaging}, 
Author = {Foose, D. P.}, 
Year = {2015}, 
Publisher = {Wright State
University} 
Address = {Dayton, OH, USA} 
Url ={http://wright.edu/~foose.3/Vespucci} 
} 
\end{verbatim} 
This citation is given in
ACS format as citation 1. \newpage \tableofcontents

%About Vespucci%

\newpage \section{About Vespucci} Vespucci is a software application developed
for imaging and analysis of hyperspectral datasets. Vespucci offers several
advantages over other software packages, including a simple, easily learned user
interface, no cost, and less restrictive licensing. Vespucci expands several
analysis techniques including univariate imaging, principal components analysis,
partial-least-squares regression, and vertex components analysis with endmember
extraction, and k-means clustering. Additionally, Vespucci can perform a number
of useful data-processing operations, including filtering, normalization,
baseline correction, and background subtraction. Datasets that consist of
spatial or temporal data with a corresponding digital signal, including
spectroscopic images, mass spectrometric images, and X-ray diffraction data can
be processed in this software. Vespucci is written in C++ and makes use of the
MLPACK, Armadillo , Qt, and QCustomPlot libraries. Vespucci is a
graphically-driven package that is designed with ease-of-use in mind and is
equally capable to other available tools. Vespucci’s capabilities are extended
by interfaces to Octave and R to allow existing research code to be run from a
common environment. Additionally, Vespucci’s C++ classes can be used to
construct more specialized programs when an application programming interface
(API) is desired. The source code and a Windows binary distribution can be
accessed at https://github.com/dpfoose/Vespucci.

Vespucci is designed with several main goals in mind: \begin{itemize} \item
Mathematical transparency. All operations are logged and reference internal
function names. The code is entirely opensource so users may examine the
internals of the program \item Ease of use. The Vespucci UI is designed to be as
easy to use as possible to remove barriers between researchers and results.
\item Extensibility. Vespucci uses the Armadillo and MLPACK libraries, which
have very intuitive APIs for novice C++ programmers familiar with MATLAB or
Octave. This makes modification of the software and contribution to its
development fairly easy. Vespucci also includes external code interfaces for R
and Octave, allowing for the use of existing research code. \item Versatility.
Vespucci can theoretically handle any dataset consisting of spatial data
combined with a corresponding digital signal. \end{itemize}


%Setup% \newpage \section{Setup} Vespucci is easy to set-up on three major
operating systems.

\subsection{Windows} Portable packages for Windows are provided and updated
regularly. These packages include all files necessary to run the program. This
version is built as an x86-64 binary (executable on just about any computer with
64-bit Windows) and uses the OpenBLAS library (which is included in the package)
for some matrix computations. To use other LAPACK/BLAS replacements in Windows,
you must compile the program from source.

\subsection{Mac OS X} Builds for Mac are periodically released. These consist of
a .app file which can be installed by dragging the file to the "Applications"
folder. The Mac versioin relies on Apple's Accelerate framework for BLAS and
LAPACK routines. The Mac version of the software is

\subsection{GNU/Linux} GNU/Linux binaries may be periodically released.
\subsection{Compiling from Source} Vespucci uses Qt's QMake build system. These
relases are updated to reflect the research needs of my group. Compiling
Vespucci requires Qt, Armadillo, LAPACK, and BLAS (or a replacement for LAPACK
and BLAS such as OpenBLAS). The MinGW\textunderscore libs and Mac\textunderscore
libs branches, when checked-out with a source branch, should make compilation
straightforward. If you are building Vespucci from source, you probably
understand how this works. Vespucci consists of three separate subprojects which
must be compiled. The Vespucci library is compiled first. The Vespucci GUI
program links to this library. The Vespucci Octave interface is compiled
separately by the mkoctfile system in Octave.


%Datasets% \section{Datasets} Vespucci can import hyperspectral datasets in
several formats: long text, wide text, and Vespucci binary (which is an
Armadillo binary file format).

\subsection{Dataset Format} Internally, Vespucci datasets are stored as
\texttt{VespucciDataset} objects. The data is stored in this object as two
spatial column vectors, \texttt{x\_} and \texttt{y\_}, a spectral abscissa
vector \texttt{abscissa\_}, and a spectral data matrix \texttt{spectra\_}. Each
column of \texttt{spectra\_} represents a spectrum, where each element is the
intensity (or absorbance, or transmittance, etc.) at the spectral abscissa value
at the same index in \texttt{abscissa\_}. The values of \texttt{x\_} and
\texttt{y\_} at that row index correspond to the spatial position at which the
spectrum was recorded. The spectral data matrix is an \(m \times n\) matrix
which can be viewed as \(n\) observations of \(m\) variables for the purpose of
multivariate analysis (this is in contrast to the row-major nature of matrices
in R).

\subsection{Importing Datasets} To import a dataset, select "Import Dataset from
File" from the "File" menu of the main window. Several options are provided. The
axis labels for the spectra and their units may be changed in this screen. These
values will be displayed in the spectra viewer.

\subsubsection{From Text Files} Text files in two formats may be imported. In
the "wide" format each row represents a spectrum. The first two columns are the
values for the two spatial coordinates. The first row represents the spectral
abscissa (wavelength or wavenumber). In the "long" format, the data is stored in
four columns. The first two columns store the spatial variables. The third
column stores a single spectral abscissa key value. The fourth column is the
value (intensity, transmittance, absorbance, etc.). Some spectrometers utilizing
the "wide" format (such as some Horiba Jobin Yvon instruments) will store y in
the first column and x in the second column. This can be accounted for by
checking the box "Swap x and y?" in the data import view.

\subsubsection{From Previously Saved Vespucci Objects} Vespucci can export
processed datasets using the Armadillo binary format. These are stored is
\texttt{arma::field<arma::mat>} objects and take the extension \emph{*.vds}.
Previously saved datasets may be imported.

\subsubsection{From Individual Point Spectra} Vespucci can create datasets from
individual point spectra

\subsection{Saving Dataset Elements} The following dataset elements can be saved
from the main window ("Export Dataset Elements" from the "Data" menu):
\begin{itemize} \item Spectra \item Average Spectra \item Average Spectra (with
abscissa) \item All Data \end{itemize} Elements may be saved as CSV or
tab-delimited ASCII for viewing or processing in other software, or as Armadillo
binary to be used by Vespucci for other purposes (such as background
subtraction). Other dataset elements can be saved using the Data Viewer ("View
Dataset Elements" in the Data menu). Larger dataset elements which may not be
displayed in the Data Viewer can be saved by selecting "Large Matrices..." from
the

%Data Pre-Processing% \newpage \section{Data Pre-Processing} Vespucci provides a
variety of basic data processing methods.

\subsection{Background Subtraction} Background subtraction subtracts a vector of
the same dimensions as the spectra from every spectra in the map. The input must
be in the armadillo binary format, but use of text format will be available in a
later release.

\subsection{Baseline Correction} Baseline correction estimates the baseline of
the spectra and subtracts this baseline from every spectra. Vespucci supports
median filter background correction and improved modified polynomial fitting
(IModPoly or the "Vancouver Raman Algorithm").

\subsection{Filtering} Filtering reduces spectral noise. Several methods are
available: Moving Average, Median, Savitzky-Golay Smoothing and SVD. Support for
median filtering is currently experimental and seems to be dependent on which
compiler is used. With the exception of SVD, all filtering methdos rely on
taking some value of a "window", a subset of the spectral signal of a given
size, centered on the value that will contain the filtered data.

\subsubsection{Median Filtering} The value of a filtered data point is equal to
the median of the window. Window size is the only adjustable parameter.
\subsubsection{Moving Average Filtering} The value of the filtered data is equal
to the average value of the window. Window size is the only adjustable
parameter. 

\subsubsection{Savitzky-Golay Smoothing} The value of the filtered
data is equal to the value of the fit of a polynomial function to the window.
The derivative of this polynomial can also be taken. Adjustable parameters are
the window size, the polynomial order and the derivative order.

\subsubsection{Truncated Singular Value Decomposition} A \emph{singular value
decomposition (SVD)} is a factorization of a matrix \(M\) such that
\(M=U{\Sigma}V'\). The diagonal entries of \({\Sigma}\) are referred to as
\emph{singular values}. A \emph{truncated} SVD is an SVD where only the \emph{k}
highest magnitude singular values are calculated. The highest magnitude singular
values are responsible for the highest amount of variance in the matrix. The
matrix can then be reconstructed from the truncated SVD, maintaining most of its
variance while removing noise. This is especially useful for relatively discrete
noise such as cosmic ray spikes. This procedure reduces the rank of the spectra
matrix to the number of singular values selected. 

\subsubsection{QUIC-SVD} The
QUIC-SVD method performs a SVD, but with the intent of maintaining a certain
amount of variance rather than maintaining a certain number of singular values.

\subsection{Normalization} Normalization manipulates the spectra so that the sum
of the intensities of the spectra equal some value. This is achieved by dividing each
value of each spectra by a weight:
\[
y_{i, \tilde{\nu}} = \frac{x_{i, \tilde{\nu}}}{w_i}
\]


\subsubsection{Unit Area Normalization} 
\emph{Unit area normalization} divides
every point of each spectra by the sum of all the points of that spectrum,
resulting in a plot where the area under the curve is 1 for every spectra. This
removes information about the difference in total intensity between spatial
points.

\subsubsection{Min/Max Normalization} 
\emph{Min/Max Normalization} subtracts the smallest value from every
element of the \texttt{spectra\_} matrix, then uses the overall maximum of each spectra as weights. 
This results in a range of spectral intensities from 0 to 1. This retains all information about the spectra except for the
absolute intensities.

\subsection{Peak Intensity Normalization} 
\emph{Peak intensity normalization} divides all points in each spectra by
the maximum value in a particular wavelength range for that spectrum so that the value
at that peak is 1.

\subsubsection{Vector Normalization} 
\emph{Vector normalization} uses the Euclidean norm of each spectra vector as weights.

\subsubsection{Standard Normal Variate} 
\emph{Standard normal variate normalization} scales the spectra by the standard
deviation so that each spectrum has a unit standard deviation:
\[w_i = \sigma_{i} + \delta\]
This scaling may also be offset by a user-specified value (\(\delta\)) which is reflective of instrument noise.

\subsubsection{Z-score Normalization}
\emph{Z-score normalization}, similarly to standard normal variate normalization, scales
each spectrum by its standard deviation. Additionally, Z-score normalization centers
each value by the mean of its spectrum. A Z-score: 
\[Z_{i, \tilde{\nu}} = \frac{x_{i, \tilde{\nu}} - \bar{x}_{i}}{\sigma_{i}}\] 
is calculated for each wavelength or wavenumber \(\lambda\) of each of the \(i\) spectra 
in the dataset, and replaces that point in the dataset with the Z-score of that 
point relative to the spectrum. If the
instrument noise is estimated, an offset may be added to the standard deviation:
\[Z_{i, \tilde{\nu}} = \frac{x_{i, \tilde{\nu}} - \bar{x}_{i}}{\sigma_{i}+\delta}\]

\subsubsection{Absolute Value Normalization}
\emph{Absolute value normalization} replaces every value with its absolute value.

\subsubsection{Spectra Scaling}
\emph{Scaling} uses a user-specified value as the weight for all spectra. This number
 may be positive or negative. This can be used to reflect the spectra about the abscissa.

\subsubsection{Mean Centering}
\emph{Mean centering} subtracts from each value the mean of all values in the spectra 
matrix at that value's row index (abscissa value):
\[x_{i, \tilde{\nu}} = x_{i, \tilde{\nu}} - \mu_{\tilde{\nu}} \]
This is commonly used before principal component analysis.

%Univariate Analysis% \newpage \section{Univariate Imaging}
\subsection{Univariate Peak Determination} Vespucci has four methods for
univariate peak determination, intensity, bandwidth, area, and derivative.

\subsubsection{Intensity} The intensity method calculates the local maximum
within the region of interest.

\subsubsection{Bandwidth} The bandwidth method calculates the full width at half
maximum (FWHM) of the curve in the region of interest relative to the local
baseline, defined as a straight line through the points on the left and right
boundaries of the region of interest. The region of interest is used only to
find a local maximum of the spectrum; bandwidths may be wider than the region of
interest. The bandwidth of a peak can be viewed as a measure of its convolution.
This may be useful in estimating the relative composition of a material where
characteristic peaks of two different materials overlap.

\subsubsection{Area} The area method determines the score for a region of
interest by calculating the area between the curve in that region and the local
baseline. The fastest method (Riemann sum) takes the sum of the differences
between the points in that region and the local baseline, defined as a straight
line through the points on the left and right boundaries of the region of
interest. This sum is multiplied by the distance between points (in spectral
abcissa units), to give the area under the curve. Peak fitting may be availible
in later releases.

\subsubsection{Derivative} The derivative method detects the edges of the peaks
close to the peak region.

%Multivarite Analysis% 
\newpage 
\section{Multivariate Imaging and Analysis}
Vespucci is capable of several multivariate analysis techniques. These methods
reduce the variables in the data set from many (the multiple values of the
spectral abcissa) to several (the principal components). Principal components
analysis (PCA), vertex component analysis (VCA) and partial least-squares (PLS)
regression reduce the total number of variables for each data point. Crisp
clustering techniques order each spectra into one of a few discrete categories
(clusters), essentially reducing the dataset to 3 dimensions (x, y, and cluster
number).

These can be performed as imaging operations (Map->New Multivariate Map) or as
analysis operations that do not produce maps (Data->Multivariate Analysis).

\subsection{Principal Component Analysis} PCA is performed using a complete
singular value decomposition. This involves finding the eigenvalues of a large
sparse matrix formed by the matrix and its transpose, then a large matrix
multiplcation to find principal components. For large matrices, this process is
computationally expensive. For this reason, the program may appear to freeze
during this process. On a 2 GHz Intel Core 2 Duo with 4 GB RAM on Windows 8.1,
this process takes about a minute for a 3524\(\times\)1694 matrix. Vespucci uses
the Armadillo function \texttt{arma::princomp} to perform PCA.

\subsubsection{Imaging} PCA images are formed from the component coefficients.
The component number is selected. Areas where that component is most responsible
for the result will be lightest in color.

\subsubsection{Data Views} The projected data, principal component coefficients,
covariance matrix eigenvalues and \(t^2\) values may be viewed and saved from
the Data Viewer (Data->View Dataset Elements). This data can be saved as
tab-delimited text, which is easily imported in MATLAB or comma-delimited text
(CSV), which can be opened in Excel. The ability to save HDF5 objects may be
availible in later releases.

\subsection{Vertex Component Analysis} VCA reduces the spectra matrix to a
particular number of components referred to as \emph{endmembers}
\cite{Nascimento2005}. Each of these endmembers represents a particular spectrum
that is responsible for a certain amount of variance in the dataset, usually
larger than 95%. VCA results are generally similar to PLS and PCA results.
Vespucci's implementation of this algorithm is found in
\texttt{arma\textunderscore ext::VCA()}.

\subsubsection{Imaging} VCA images are based on the extent to which an endmember
is responsible for the spectrum.

\subsubsection{Data Views} Plots of endmembers may be viewed and saved from the
Data Viewer (Data->View Dataset Elements). Statistics related to each endmember
may also be saved.

\subsection{Partial Least Squares Regression} PLS is another method for reducing
the number of variables in the spectra matrix. PLS provides similar results to
PCA, but with less computational expense. To maintain similarity to the MATLAB
code used previously by my group, PLS is implemented using the SIMPLS algorithm
\cite{deJong1993}. Vespucci's implementation of this algorithm is found in
\texttt{arma\textunderscore ext::plsregres()}.

\subsubsection{Imaging} PLS Images, like PCA images, are formed based on the
coefficient.

\subsubsection{Data Views} PLS statistics may be viewed and saved from the Data
Viewer (Data->View Dataset Elements).

\subsection{Crisp Clustering}

\subsubsection{\emph{k}-Means Clustering} \emph{k}-Means clustering orders every
spectra into one of several categories based on the \emph{k}-nearest neighbors
algorithm. Vespucci perfoms this using \texttt{mlpack::kmeans::Kmeans}.



%Viewing Images and Spectra % 
\newpage 
\section{Viewing and Saving Images and Spectra} 
Images can be saved in a variety of formats (File->Save Image As from
the map viewer). Double-clicking on a map name in the map list will open or
close the map window. \subsection{The Map Viewer} \subsubsection{The Display
Menu} Images created by Vespucci are created to fairly arbitrary pre-defined
size (designed to make a 50 by 50 map fit in the middle third of a 1280x720
screen. Normally, resizing the window is locked. Deselecting the "Lock Size"
option allows the window to be resized. To make the window size proportional to
the spatial data, select the "Reproportion" option.

Image interpolation will smooth the edges of the points, producing a more
visually appealing image. However, this interpolation may create a false sense
of increased resolution.

The axes and color scale can be enabled or disabled.

The map can be attached to the global color scale.

\subsubsection{The Data Menu} A new dataset can be produced from a map by
selecting (Data->New Dataset from Map...). The new dataset will contain only
those spectra with a map value between the two bounds specified.
\subsubsection{The Spectrum Viewer} When the Spectrum Viewer is open
(Display->Spectrum Viewer), the spectrum at a particular point in the map may be
viewed by clicking on the map. The bottom left corner displays the spatial data
as an X-Y pair. The bottom center displays the value of the map at that point.
The Export button can be used to save the image of the spectrum or the spectrum
itself as CSV, text, or binary. \subsubsection{Saving Images} Images are saved
using (File->Save Image As...). Everything visible in the Map Viewer will be
saved.




\bibliography{references} \end{document}